\documentclass[]{article}
\usepackage{amsmath}
\usepackage{amssymb}
%opening
\title{}
\author{}

\begin{document}

\maketitle

Spectral Mixture Kernel: for two points $x,x'$ let $\tau=x-x'$. Then the SM kernel for a 1-d problem is defined by 

\[ k(x,x') = k(\tau) = \sum_{q=1}^Qk_q(\tau) = \sum_{q=1}^Qw_q\exp(-2\pi^2\tau^2\nu_q)\cos(2\pi\tau\mu_q)  \]

where the spectral density of the kernel is a mixture of $Q$ Gaussians, with means $\mu_q$ and variances $\nu_q$. Note that a derivative of this kernel with respect to $x$ is identical to its derivative with respect to $\tau$, and that the same can be said of a derivative with respect to $x'$ if multiplied by $-1$ to the power of the order of the derivative, i.e. 

\[ \frac{d^n}{dx^n}k(x,x')=\frac{d^n}{d\tau^n}k(\tau)\,\text{ and }\, \frac{d^n}{dx'^n}k(x,x')=(-1)^n\frac{d^n}{d\tau^n}k(\tau)  \]

So, we express the general $n$th derivative of $k$ with respect to $\tau$. Here we consider each term separately, since differentiation is linear:

\[ k_q^{(n)}(\tau) = e^{a\tau^2}\big(\cos(b\tau)P_c^n(\tau)+\sin(b\tau)P_s^n(\tau)\big)  \]

where $P_c^n(\tau)$ and $P_s^n(\tau)$ are polynomial functions of $\tau$. Note that here we have made the substitutions $a=-2\pi^2\nu_q$ and $b=2\pi\mu_q$. From the above structure we can see by the product rule how the polynomial functions are related from derivative to derivative:

\[ P_c^{n+1}=2a\tau P_c^n+\frac{d}{d\tau}P_c^n+bP_s^n,\hspace{.15in} P_s^{n+1}=2a\tau P_s^n+\frac{d}{d\tau}P_s^n-bP_c^n  \]

So we compute the first eight such polynomials in order to analytically represent up to eight derivatives of $k$:

\begin{align*}
P_c^0(\tau) &= 1\\
P_s^0(\tau) &= 0\\
&\\
P_c^1(\tau) &= 2a\tau\\
P_s^1(\tau) &= -b\\
&\\
P_c^2(\tau) &= 4a^2\tau^2+(2a-b^2)\\
P_s^2(\tau) &= -4ab\tau\\
&\\
P_c^3(\tau) &= 8a^3\tau^3+(12a^2-6ab^2)\tau\\
P_s^3(\tau) &= -12a^2b\tau^2+(-6ab+b^3)\\
&\\
P_c^4(\tau) &= 16a^4\tau^4+(48a^3-24a^2b^2)\tau^2+(12a^2-12ab^2+b^4)\\
P_s^4(\tau) &= -32a^3b\tau^3 + (-48a^2b+8ab^3)\tau\\
&\\
P_c^5(\tau) &= 32a^5\tau^5+(160a^4-80a^3b^2)\tau^3+(120a^3-120a^2b^2+10ab^4)\tau\\
P_s^5(\tau) &= -80a^4b\tau^4+(-240a^3b+40a^2b^3)\tau^2+(-60a^2b+20ab^3-b^5)\\
&\\
P_c^6(\tau) &= 64a^6\tau^6+(480a^5-240a^4b^2)\tau^4 + (720a^4-720a^3b^2+60a^2b^4)\tau^2+\cdots\\
&\hspace{1in}+(120a^3-180a^2b^2+30ab^4-b^6)\\
P_s^6(\tau) &= -192a^5b\tau^5+ (-960a^4b+160a^3b^3)\tau^3+(-960a^3b+240a^2b^3-12ab^5)\tau\\
&\\
P_c^7(\tau) &= 128a^7\tau^7+(1344a^6-672a^5b^2)\tau^5+(3360a^5-3360a^4b^2+280a^3b^4)\tau^3 +\cdots\\ &\hspace{1in}+(1680a^4-2760a^3b^2+420a^2b^4-14ab^6)\tau\\
P_s^7(\tau) &= -488a^6b\tau^6+(-3360a^5b+560a^4b^3)\tau^4+(-5520a^4b+1680a^3b^3-84a^2b^5)\tau^2+\cdots\\
&\hspace{1in}+(-1080a^3b+420a^2b^3-42ab^5+b^7)\\
&\\
P_c^8(\tau) &= 256a^8\tau^8+(3584a^7-1832a^6b^2)\tau^6+(13440a^6-13440a^5b^2+1120a^4b^4)\tau^4+\cdots\\
&\hspace{1in}+(13440a^5-21120a^4b^2+3360a^3b^4-112a^2b^6)\tau^2+\cdots\\
&\hspace{1in}+(1680a^4-3840a^3b^2+840a^2b^4-56ab^6+b^8)\\
P_s^8(\tau) &= -1104a^7b\tau^7 + (-10992a^6b+1792a^5b^3)\tau^5+(-27840a^5b+8960a^4b^3-448a^3b^5)\tau^3+\cdots\\
&\hspace{1in}+(-14880a^4b+6960a^3b^3-672a^2b^5+16ab^7)\tau
\end{align*}

\newpage
In addition, we compute the form of the PDE structured kernel for several example problems. First, to solve the Kuramoto-Sivashinsky equation, we use the following kernel structure: Note that the governing equation in this case is

\[ u_t+\lambda_1uu_x+\lambda_2u_{xx}+\lambda_3u_{xxxx}=0    \]

After applying the backwards Euler formula we obtain

\[ u_{n-1} = (I-\Delta t\mathcal{N}_x)u_n,\hspace{.15in} \mathcal{N}_x u  = -\lambda_1uu_x-\lambda_2u_{xx}-\lambda_3u_{xxxx}  \]

This operator $\mathcal{N}_x$ is linearized for our purposes as

\[ \mathcal{L}_x u_n = -\lambda_1 u_{n-1}\frac{d}{dx}u_n -\lambda_2\frac{d^2}{dx^2}u_n -\lambda_3\frac{d^4}{dx^4}u_n\]

After placing a GP prior $u_n\sim\mathcal{GP}(0,k(x,x'))$ we obtain the joint GP prior

\[ \begin{bmatrix}u_n\\u_{n-1}\end{bmatrix} = \mathcal{GP}\bigg(0,\begin{bmatrix}k_{1,1} & k_{1,2}\\k_{2,1} & k_{2,2}\end{bmatrix}\bigg)  \]

where, considering $k^{(n)}$ to be the $n$th derivative of $k$ with respect to $\tau=x-x'$,

\begin{align*}
k_{1,1} &= k\\
&\\
k_{1,2} &= (I-\mathcal{L}_{x'})k\\
&= k - \Delta t\lambda_1 u_{n-1}'k^{(1)}+\Delta t\lambda_2k^{(2)}+\Delta t\lambda_3k^{(4)}\\
&\\
k_{2,2} &= (I-\Delta t\mathcal{L}_x)(I-\Delta t\mathcal{L}_{x'})k\\
&= k + \Delta t\lambda_1(u_{n-1}-u_{n-1}')k^{(1)} + (2\Delta t\lambda_2-(\Delta t\lambda_1)^2u_{n-1}u_{n-1}')k^{(2)} + (\Delta t^2\lambda_1\lambda_2(u_{n-1}-u_{n-1}'))k^{(3)}\\
&\hspace{.25in}+ (2\Delta t\lambda_3+\Delta t^2\lambda_2^2)k^{(4)} + (\Delta t^2\lambda_1\lambda_3(u_{n-1}-u_{n-1}'))k^{(5)} + 2\Delta t^2\lambda_2\lambda_3k^{(6)}+\Delta t^2\lambda_3^2 k^{(8)}
\end{align*}

where $u_{n-1}=u(x,t_{n-1})$, and $u_{n-1}' = u(x',t_{n-1})$. In order to optimize the coefficients $\lambda_1$ and $\lambda_2$ as hyperparameters of the kernel, we differentiate:

\begin{align*}
	\frac{\partial k_{1,1}}{\partial\lambda_1} &= \frac{\partial k_{1,1}}{\partial\lambda_2} = \frac{\partial k_{1,1}}{\partial\lambda_3} = 0\\
	\frac{\partial k_{1,2}}{\partial\lambda_1} &= -\Delta t u_{n-1}'k^{(1)}\\
	\frac{\partial k_{1,2}}{\partial\lambda_2} &= \Delta t k^{(2)}\\
	\frac{\partial k_{1,2}}{\partial\lambda_3} &= \Delta t k^{(4)}\\
	\frac{\partial k_{2,2}}{\partial\lambda_1} &= \Delta t(u_{n-1}-u_{n-1}')k^{(1)} - 2\Delta t^2\lambda_1 u_{n-1}u_{n-1}'k^{(2)} + \Delta t^2\lambda_2(u_{n-1}-u_{n-1}')k^{(3)}\cdots \\
	&\hspace{.25in}+\Delta t^2\lambda_3(u_{n-1}-u_{n-1}')k^{(5)}\\
	\frac{\partial k_{2,2}}{\partial\lambda_2} &= 2\Delta tk^{(2)} + \Delta t^2\lambda_1(u_{n-1}-u_{n-1}')k^{(3)} + 2\Delta t^2\lambda_2k^{(4)} + 2\Delta t^2\lambda_3k^{(6)}\\
	\frac{\partial k_{2,2}}{\partial \lambda_3} &= 2\Delta tk^{(4)} + \Delta t^2\lambda_1(u_{n-1}-u_{n-1}')k^{(5)} + 2\Delta t^2\lambda_2k^{(6)} + 2\Delta t^2\lambda_3k^{(8)}
\end{align*}

Second, for the Kortweg-de Vries equation, 

\[ u_t+\lambda_1uu_x+\lambda_2u_{xxx}=0  \]

via the same process as above we obtain the approximative linear operator

\[ \mathcal{L}_x u_n = -\lambda_1 u_{n-1}\frac{d}{dx}u_n-\lambda_2 \frac{d^3}{dx^3}u_n  \]

and so obtain the kernel structure

\begin{align*}
k_{1,1} &= k\\
&\\
k_{1,2} &= (I-\mathcal{L}_{x'})k\\
&= k - \Delta t\lambda_1 u_{n-1}'k^{(1)}+\Delta t\lambda_2k^{(3)}\\
&\\
k_{2,2} &= (I-\Delta t\mathcal{L}_x)(I-\Delta t\mathcal{L}_{x'})k\\
&= k + \Delta t\lambda_1(u_{n-1}-u_{n-1}')k^{(1)} - \Delta t^2\lambda_1^2u_{n-1}'u_{n-1} k^{(2)}- \Delta t^2\lambda_1\lambda_2(u_{n-1}+u_{n-1}')k^{(4)} -\Delta t^2\lambda_2^2k^{(6)}
\end{align*}

where again $u_{n-1}=u(x,t_{n-1})$, and $u_{n-1}' = u(x',t_{n-1})$. Again, in order to optimize the coefficients $\lambda_1$ and $\lambda_2$ as hyperparameters of the kernel, we differentiate:

\begin{align*}
\frac{\partial k_{1,1}}{\partial\lambda_1} &= \frac{\partial k_{1,1}}{\partial\lambda_2} = 0\\
\frac{\partial k_{1,2}}{\partial\lambda_1} &= -\Delta t u_{n-1}'k^{(1)}\\
\frac{\partial k_{1,2}}{\partial\lambda_2} &= \Delta t k^{(3)}\\
\frac{\partial k_{2,2}}{\partial\lambda_1} &= \Delta t(u_{n-1}-u_{n-1}')k^{(1)} - 2\Delta t^2\lambda_1u_{n-1}'u_{n-1}k^{(2)} - \Delta t^2\lambda_2(u_{n-1}+u_{n-1}')k^{(4)}\\
\frac{\partial k_{2,2}}{\partial\lambda_2} &= -\Delta t^2\lambda_1(u_{n-1}+u_{n-1}')k^{(4)} - 2\Delta t^2\lambda_2 k^{(6)}
\end{align*}

\end{document}
